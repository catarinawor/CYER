 \documentclass[11pt,a4paper]{article}

%% Compile this with
%% R> library(pgfSweave)
%% R> pgfSweave
%\usepackage[nogin]{Sweave} %way to understand R code
\usepackage{pgf} 

%look it up
\usepackage{tikz} 

% allows R code or other stuff in the figure labels
\usepackage{subfigure} 

\usepackage{color,colortbl}

\usepackage[left=1.2in,right=1.2in,top=1.2in,bottom=1.2in]{geometry} 

% sets the margins
\usepackage{fancyhdr} 
\usepackage{setspace} 
\usepackage{indentfirst} 
\usepackage{titlesec} 
\usepackage{natbib} 
\usepackage{sectsty}
\usepackage{listings}
\usepackage[section]{placeins}

\usepackage{amssymb}
\usepackage{amsmath}
\usepackage{amsfonts}
\usepackage{float}
\usepackage[hyphens,obeyspaces]{url}


\usepackage{lmodern}
\usepackage[T1]{fontenc}
\usepackage{textcomp}
\usepackage{longtable}
\usepackage{pdflscape}

\newcommand{\lang}{\textsf} 

%
\newcommand{\code}{\texttt} 
\newcommand{\pkg}{\texttt} 
\newcommand{\ques}[1]{{\bf\large#1}} 
\newcommand{\eb}{\\
\nonumber} 


%% The following is used for tables of equations.
\newcounter{saveEq} 
\def\putEq{\setcounter{saveEq}{\value{equation}}} 
\def\getEq{\setcounter{equation}{\value{saveEq}}} 
\def 
\tableEq{ 

% equations in tables
\putEq \setcounter{equation}{0} 
\renewcommand{\theequation}{T\arabic{table}.\arabic{equation}} \vspace{-5mm} } 
\expandafter\def\expandafter\UrlBreaks\expandafter{\UrlBreaks%  save the current one
  \do\a\do\b\do\c\do\d\do\e\do\f\do\g\do\h\do\i\do\j%
  \do\k\do\l\do\m\do\n\do\o\do\p\do\q\do\r\do\s\do\t%
  \do\u\do\v\do\w\do\x\do\y\do\z\do\A\do\B\do\C\do\D%
  \do\E\do\F\do\G\do\H\do\I\do\J\do\K\do\L\do\M\do\N%
  \do\O\do\P\do\Q\do\R\do\S\do\T\do\U\do\V\do\W\do\X%
  \do\Y\do\Z}


\def\normalEq{ 

% renew normal equations
\getEq 
\renewcommand{\theequation}{\arabic{section}.\arabic{equation}}}
\newcommand{\normal}[2]{\ensuremath{N(#1,#2)}}


\onehalfspacing 
\definecolor{Gray}{gray}{0.9}

\title{Calculation of Calendar Year Exploitation rates and simulation based variances}
\author{Catarina Wor} 
\sectionfont{
\fontsize{12}{12}\selectfont}
\lstset{ language={[LaTeX]TeX},
      escapeinside={{(*@}{@*)}}, 
       gobble=0,
       stepnumber=1,numbersep=5pt, 
       numberstyle={\footnotesize\color{gray}},%firstnumber=last,
      breaklines=true,
      framesep=5pt,
      basicstyle=\small\ttfamily,
      showstringspaces=false,
      keywordstyle=\ttfamily\textcolor{blue},
      stringstyle=\color{orange},
      commentstyle=\color{black},
      rulecolor=\color{gray!10},
      breakatwhitespace=true,
      showspaces=false,  % shows spacing symbol
      backgroundcolor=\color{gray!15}}


%\pgfrealjobname{pgfSweave-vignette}
%pgfSweave-vignette
\begin{document}
\maketitle

%\tableofcontents


\section{Background}

%The calendar year exploitation rates (CYER) are the base quantitity to determine exploitation rate targets for chinook salmon stocks.


We calculate the calendar year exploitation rates (CYERs) and associated variance estimates using analytical and simulation based methods. The analytical methods are described by \citet{bernard_estimating_1996} and extended by the \citet{pacific_salmon_commission_coded_wire_tag_workgroup_action_2008}. For the purposes of this analysis, we limited the analysis to the the Harrison Chinook stock, for the BC fisheries, and in the years between 2009 and 2015, only.


\section{Data processing}

All the data described below were pasted as values in excel and saved into .csv files. Subsequently the data was imported into R for processing. Datasets for escapement and recoveries were combined for calculation of exploitation rates. 
%I tried using the function ``read_excel from the ``readxl'' package but I kept running into problems with R erroneously interpreting numerical values as factors. 


\subsection{Escapement data}

Escapement data was provided by Chuck Parken in a spreadsheet named \path{Harrison River Escapement CWT Data 1984-2017_16Jul2018.xlsx}. Most of the data used in this analysis was extracted from the spreadsheet \path{Harrison - Escapement CWT Data}. Including data on escapement estimates, tag recoveries, tag decoding status, tag codes, adipose fin clipped status and other. 


%A detailed description of all the data in the \path{Harrison - Escapement CWT Data} spreadsheet is given in \path{HarrisonEscapementCWTData.pdf} and a

An overview of the data collection methods used to produce the escapement data is given in \citet{parken_estimation_2006}. Escapement estimates are segregated by sex because of varying sampling probabilities experienced by fish of different sizes, with jacks having much lower sampling rates than other groups. The spreadsheet also contained additional information on Peterson tagging experiments, proportions of adipose fin clipped fish and and coded-wire tags recoveries. One correction was made to this spreadsheet: In 2013, the spawner estimate was inconsistent across tag codes, with the ``no data'' category indicating that one additional fish was present. This difference was ignored and the value in this row was set equal to the other rows for the purposes of this analysis. 

 This excel file also contained information regarding the CWT codes used for Canadian stocks in spreadsheet \path{Cdn CWTs for ERA in 2018}. This information was used to determine fish age, stock and release years for all recoveries in the fisheries and on the spawning grounds. 



\subsection{Catch data}

The catch and recovery data were extracted from the MRP program database. Catches might not reflect complete catches, but instead are supposed to represent the catches in which CWT tags were recovered (more details: Nick Komick, MRP program) 
The total catch and fisheries recoveries data were processed following a few steps:
%The \path{Harrison Recs (2009 to 2015)} spreadsheet was combined with the information from \path{Cdn CWTs for ERA in 2018} to generate tag specific information regarding the release year, stock, and age of fish at capture. 

\begin{itemize}
  \item Combine the tag recovery information with CWT code details to determine the release year, stock, and age of fish at capture.
  \item Exclude any data for which there were no identifying variable (i.e., \path{STS_ID}) or sample rate information (i.e., \path{CWT_estimate} is missing). These are data uncertain sampling rates or recoveries that were reported outside of the MRP program. 
  \item Aggregate number of tags recovered by unique \path{STS_ID}, year, fishery and period of the year.
  \item Combine data from recoveries and total catches by \path{STS_ID},i.e., match recovered tags to the fisheries catch. 
  \item Remove data from all fisheries that did not capture any (Harrison) tags. 
\end{itemize}

In addition, the catch data was reclassified into Individual Stock Based Management (ISBM) and Aggregate Abundance based management (AABM) fisheries. This grouping was made based on spatiotemporal information available for most catches. Uncertainty exists, however, regarding some of the catch that was reported for the aggregate "West Coast of Vancouver Island" (WCVI). The WCVI catch can be either classified as AABM or ISBM depending on exact location and time of the year as defined in the Pacific Salmon Treaty \citep{pacific_salmon_commission_pacific_2014}. Unfortunately, the data available to us at this point did not include the detail necessary to appropriately split those catches. For the purposes of this preliminary analysis, all the WCVI catch was grouped under AABM.  Future iterations of this work should include finer catch resolution and allow for the appropriate differentiation of WCVI AABM and ISBM fisheries. 


\section{CYER and Uncertainty calculations}


\subsection{Analytical approach}

The analytical approach relied on the methods described by \citet{bernard_estimating_1996} and \citet{pacific_salmon_commission_coded_wire_tag_workgroup_action_2008} - eqs 5.5 and 5.6. The model equations and variable definitions are reproduced in this section. We also updated and standardized model notation. We use the notation described for a wild stock caught in recreational fisheries because these formulation collapse to the other fisheries/stocks categories when the variance components associate with $G(N_{i})$, $G(p_{ijy})$ and $G(\theta_{j})$ are set to zero. 

\begin{table}
  \begin{center}
    \caption{Variable definition.}
    \label{tab:table1}
    \begin{tabular}{l p{6.cm} c} % <-- Alignments: 1st column left, 2nd middle and 3rd right, with vertical lines in between
     Notation & Parameter & code/function name\\
      \hline
      \rowcolor{Gray}
      $i $& index for strata & \path{fishery}\\
      $y $& index for year & \path{Year} \\
      \rowcolor{Gray}
      $j $& index for age & \path{Age}\\
      $N_{iy}$ & catch in each strata and year &\path{Ni}\\
      \rowcolor{Gray}
      $M_{ijy}$ & number of tags recovered in each strata at each age and year& \path{obs_tag$Mij}\\
      $\lambda_{iy}$ &decoding rate of CWT in each strata and year&\path{lambdaphi$lambdai}\\
      \rowcolor{Gray}
      $\phi_{iy}$ & fraction of catch that is inspected for CWT in each strata and year&\path{lambdaphi$phi}\\
      $\theta_{j}$ & proportion of each cohort that is tagged with CWTs - assume all fish are tagged and estimate exploitation rate of tagged cohort only and year &\path{thetaj}\\
      \rowcolor{Gray}
      $G(\theta_{j})$ &squared cv of proportion of each cohort that is tagged with CWTs - set to zero &\path{Gthetaj}\\
      $G(N_{iy})$ &squared cv of the catch in each strata and year &\path{GNi}\\
      \rowcolor{Gray}
      $G(p_{ijy})$&squared cv for the probability that a fish caught in stratum i has a tag from cohort j&\\
      $r_{ijy}$&number of (tagged) fish harvested in stratum i from cohort j &\\
      \rowcolor{Gray}
      $T_{iy}$&harvest from several cohorts at stratum i&\\
      $ER_{iy}$ &exploitation rates by year and stratum&\\
      \rowcolor{Gray}
      $var(ER_{iy})$ &variance of exploitation rates by year and stratum&\\
      $95\% CI(ER_{iy})$& 95\% confidence intervals by exploitation rates&\\
      %$cv(T_{y})$&cv of annual catch - may be derived from $var(T_{iy})$ if measure across strata are independent&\\
    \end{tabular}
  \end{center}
\end{table}

%\begin{align}
%r_{ijy}&=\frac{M_{ijy}}{\lambda_{iy}\cdot \phi_{iy} \cdot \theta_{j}}
%\end{align}
Computation of exploitation rates was split into three steps: (i) computation of harvest by strata and age (\path{calc_rij()} and eqs. 1-3), (ii) computation of annual catches by strata (all ages) (\path{calc_Ti()} and eqs. 4-5), and (iii) computation of exploitation rates (\path{calc_CYER()} and eqs. 6-\ref{erci}) and confidence intervals.  For the calculation of the confidence intervals of the exploitation rates, we used the normal approximation (eq. \ref{erci}). However, this approximation allow confidence intervals to exceed 1 which is impossible in reality. 

\begin{align}
\label{rijy}
r_{ijy}&=\frac{M_{ijy}}{\lambda_{iy}\cdot \phi_{iy} \cdot \theta_{j}}\\
G(p_{ijy}) &=1-\frac{\phi_{iy}\cdot\theta_{j}}{M_{ijy}}\\
var(r_{ijy})&= r_{ijy}^2\cdot (G(p_{ijy})+G(N_{i})+G(\theta_{j})-G(\theta_{j})\cdot G(N_{iy})-G(\theta_{j})\cdot G(p_{ijy})\nonumber\\
&-G(p_{ijy})\cdot G(N_{i})+G(p_{ijy})\cdot G(N_{i})\cdot G(\theta_{j}))
\end{align}


\begin{align}
\label{Tiy}
T_{iy}&=\sum^{j}{r_{ijy}}\\
var(T_{iy})&=\sum^{j}{var(r_{ijy})}+2\cdot \sum_j\sum_{k<j}(r_{ijy}\cdot r_{iky})\cdot G(N_{iy})
\end{align}


\begin{align}
\label{ERiy}
ER_{iy}&=\frac{T_{iy}}{\sum_{i}{T_{iy}}}\\
var(ER_{iy})&={ER_{iy}}^{2} \cdot \left(\frac{var(T_{iy})}{{T_{iy}}^2} + \frac{\sum^{i}{var(T_{iy})}}{\sum^{i}{T_{iy}}^2}\right) \\
95\% CI(ER_{iy})&=ER_{iy} \pm 1.96\cdot \sqrt{var(ER_{iy})}
\label{erci}
\end{align}


\subsection{Simulation approach}

The estimates of total catches and exploitation rates are the same as the one used in the previous above (Eqs. \ref{rijy}, \ref{Tiy}, and \ref{ERiy}.  the confidence intervals, however are produced by assuming that the sampling process follows a series of statistical distributions with parameters equal to the observed values. In all the equations below the \~ is used to indicate simulated quantities

Catch error was assumed to ne log-normally distributed:

\begin{align}
\widetilde{N_{iy}} = N_{iy}\cdot e^{\varepsilon_{iy}}\\
\varepsilon_{iy}\sim \mathcal{N}\left(0,\sqrt{G(N_{iy})}\right)
\end{align}

Sampling error was assumed to follow a binomial distributtion (catches was either samples or not sampled. )

\begin{align}
\widetilde{n_{iy}} \sim  Binom(N_{iy}\cdot\phi_{iy},\phi_{iy})
\end{align}
Where $N_{iy}\cdot\phi_{iy}$ indicate the sample size and $\phi_{iy}$  is the sample rate. 


If the sample was simulated to be positive, then a hypergeometric distribution was applied to add error associated with presence of tagged fish in the sample.

Once the number of tagged fish was generated, binomial error was added to indicate if a tag was decoded or not. 


\section{Results}

\subsection{Analytical approach}
The Harrison stock CYERs were calculated based on escapement data and BC fisheries. Some of the assumptions of this analysis are listed below.

\begin{itemize}
\item Assumed that the coefficient of variation (CV) for all catch ($T_{iy}$) is set to 0.1. The true standard deviation was available for the escapement data. 
\item Assumed that the proportion of tagged fish is 1.0 and perfectly known (var=0). This implies that only tagged fish harvest is considered - Should have no impact on ER estimates for the indicator stocks.
\item The number of tags per strata and age $M_{ijy}$ was not transformed into adults equivalents. This feature is yet to be implemented. 
\end{itemize}


%In order to demonstrate the vulnerability of the CYER methods to the inclusion of different sources of data, we produced CYERs with and without the escapement data in the denominator of the exploitation rates calculations. The CYER dramatically increase when escapement data is not considered (figs. \ref{BCCYER} and \ref{BCCYERnoesc}). When escapement data is not considered, both magnitude and trends in exploitation rates are affected (figs. \ref{BCCYER} and \ref{BCCYERnoesc}). We chose 

%\begin{landscape}
\begin{figure}[htbp]
  \centering
  \includegraphics[scale=.48]{../figs/treatyBC_CYER}
  \caption{Calendar year exploitation rates (CYER) for Harrison river stock based on escapement data and BC fisheries grouped by ISBM and AABM. Confidence intervals generated by analytical solution. }
\label{BCCYER}
\end{figure}

\begin{figure}[htbp]
  \centering
  \includegraphics[scale=.48]{../figs/sim_CYER}
  \caption{Calendar year exploitation rates (CYER) for Harrison river stock based on escapement data and BC fisheries grouped by ISBM and AABM and confidence interval generated by simulation.}
\label{BCCYER}
\end{figure}


%\end{landscape}

%\begin{landscape}
%\begin{figure}[htbp]
%  \centering
%  \includegraphics[scale=.48]{../figs/no_esc_CYER}
%  \caption{Calendar year exploitation rates (CYER) for Harrison river stock based on and BC fisheries data, only}
%\label{BCCYERnoesc}
%\end{figure}
%\end{landscape}



\section{Future steps}

\begin{itemize}
\item{Retrieve real CV for catch data ({\color{red} Dawn/Nick?}) -- Already have CV for escapement data.}
\item{Check use of escapement data - Chuck}
\item{Adjust confidence intervals so that they are bound between 0 and 1. Any ideas?}
\item{Identify how the Adult Equivalent adjustment may affect affect estimates.}
\item{Discuss ways to loosen the assumptions listed above. }
\item{Simulation based confidence intervals. - will loosen assumptions and produce confidence intervals between 0 and 1. }
\item{Suggestions?}
\end{itemize}


\bibliographystyle{apa} 
\bibliography{CYER.bib}


\end{document}




